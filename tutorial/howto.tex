%%%%%%%%%%%%%%%%%%%%%%%%%%%%%%%%%%%%%%%%%%%%%%%%%%%%%%%%%%%%%%%%%%%%%%%%%%%%%%
%%%%%%%%%%%%%%%%%%%%%%%%%%%%%%%%%%%%%%%%%%%%%%%%%%%%%%%%%%%%%%%%%%%%%%%%%%%%%%
\begin{frame}[fragile=singleslide]
\mode<presentation>{\frametitle{\insertsubsection\ -- Inline Guards}}
\begin{itemize}
\item simple guards may be inlined
\item in a previous example we use a guard function ...
\begin{lstlisting}
  bool hasToken() const{
    return count<size;
  }
\end{lstlisting}
\item ... and refered to the guard inside the FSM 
\begin{lstlisting}
       GUARD(Source::hasToken)  >>
\end{lstlisting}
\item instead we can inline this simple guards
\begin{lstlisting}
      (VAR(count)<VAR(size))   >>
\end{lstlisting}
\item in a FSMs transition we can directly access values of variables using \lstinline!VAR(...)! the macro
\end{itemize}
\end{frame}





%%%%%%%%%%%%%%%%%%%%%%%%%%%%%%%%%%%%%%%%%%%%%%%%%%%%%%%%%%%%%%%%%%%%%%%%%%%%%%
%%%%%%%%%%%%%%%%%%%%%%%%%%%%%%%%%%%%%%%%%%%%%%%%%%%%%%%%%%%%%%%%%%%%%%%%%%%%%%
\begin{frame}[fragile=singleslide]
\mode<presentation>{\frametitle{\insertsubsection\ -- Use Guards for Control Flow}}
\begin{itemize}
\item model ``if/else'' like constructs
\item use \lstinline|GUARD(..)| and  \lstinline|!GUARD(..)| for different transitions
\item please note the \lstinline|!|
\begin{lstlisting}
    state = 
      GUARD(Actor::testGuard)  >>  // case IF
      out(1)                   >>
      CALL(Actor::processA)    >> state
    |
      !GUARD(Actor::testGuard) >>  // case IF NOT
      out(1)                   >>
      CALL(Actor::processB)    >> state
      ;
\end{lstlisting}
\end{itemize}
\end{frame}





%%%%%%%%%%%%%%%%%%%%%%%%%%%%%%%%%%%%%%%%%%%%%%%%%%%%%%%%%%%%%%%%%%%%%%%%%%%%%%
%%%%%%%%%%%%%%%%%%%%%%%%%%%%%%%%%%%%%%%%%%%%%%%%%%%%%%%%%%%%%%%%%%%%%%%%%%%%%%
\begin{frame}[fragile=singleslide]
\mode<presentation>{\frametitle{\insertsubsection\ -- Use Guards for Control Flow}}
\begin{itemize}
\item to be precise, the previous example is not an ``if/else'' construct
\item it is a ``if/if not'' construct
\item you cannot model a explicit ``else'' transition
\item neither an exclusive default transition
\item instead you have to model an explicit ``if not'' transition (see ``else if'' constructs on next slide)
\item transition have no priority ...
\item ``if/else if/else'' constructs in software do have a priority
\end{itemize}
\end{frame}





%%%%%%%%%%%%%%%%%%%%%%%%%%%%%%%%%%%%%%%%%%%%%%%%%%%%%%%%%%%%%%%%%%%%%%%%%%%%%%
%%%%%%%%%%%%%%%%%%%%%%%%%%%%%%%%%%%%%%%%%%%%%%%%%%%%%%%%%%%%%%%%%%%%%%%%%%%%%%
\begin{frame}[fragile=singleslide]
\mode<presentation>{\frametitle{\insertsubsection\ -- Use Guards for Control Flow}}
\begin{itemize}
\item a similary way for ``if/else if'' constructs
\begin{lstlisting}
  (GUARD(Source::testGuard1) && GUARD(Source::testGuard2))   >>
  ...
  (GUARD(Source::testGuard1) && !GUARD(Source::testGuard2))  >>
  ...
  (!GUARD(Source::testGuard1) && GUARD(Source::testGuard2))  >>
  ...
  (!GUARD(Source::testGuard1) && !GUARD(Source::testGuard2)) >>
  ...
\end{lstlisting}
\end{itemize}
\end{frame}













