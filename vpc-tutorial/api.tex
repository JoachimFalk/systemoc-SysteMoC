\lstset{language=C++}

%%%%%%%%%%%%%%%%%%%%%%%%%%%%%%%%%%%%%%%%%%%%%%%%%%%%%%%%%%%%%%%%%%%%%%%%%%%%%%
%%
%%%%%%%%%%%%%%%%%%%%%%%%%%%%%%%%%%%%%%%%%%%%%%%%%%%%%%%%%%%%%%%%%%%%%%%%%%%%%%
\subsection{Overview}

%%%%%%%%%%%%%%%%%%%%%%%%%%%%%%%%%%%%%%%%%%%%%%%%%%%%%%%%%%%%%%%%%%%%%%%%%%%%%%
%%%%%%%%%%%%%%%%%%%%%%%%%%%%%%%%%%%%%%%%%%%%%%%%%%%%%%%%%%%%%%%%%%%%%%%%%%%%%%
\begin{frame}[t]
\mode<presentation>{\frametitle{\insertsubsection\ }}

\begin{itemize}
\item The API allows the configuration of the architecture model and mappings to be programmed in C++.
\item Using the API may be used instead of a XML configuration file.
\item Note: Mixing both, API and XML configuration, is not possible.
\end{itemize}

\end{frame}

%%%%%%%%%%%%%%%%%%%%%%%%%%%%%%%%%%%%%%%%%%%%%%%%%%%%%%%%%%%%%%%%%%%%%%%%%%%%%%
%%
%%%%%%%%%%%%%%%%%%%%%%%%%%%%%%%%%%%%%%%%%%%%%%%%%%%%%%%%%%%%%%%%%%%%%%%%%%%%%%
\subsection{Example}


%%%%%%%%%%%%%%%%%%%%%%%%%%%%%%%%%%%%%%%%%%%%%%%%%%%%%%%%%%%%%%%%%%%%%%%%%%%%%%
%%%%%%%%%%%%%%%%%%%%%%%%%%%%%%%%%%%%%%%%%%%%%%%%%%%%%%%%%%%%%%%%%%%%%%%%%%%%%%
\begin{frame}[fragile=singleslide]
\mode<presentation>{\frametitle{\insertsubsection\ -- API Header}}
\begin{lstlisting}
#include <systemoc/smoc_moc.hpp>

#ifdef SYSTEMOC_ENABLE_VPC
#include <vpc_api.hpp>
#endif // SYSTEMOC_ENABLE_VPC

class Source: public smoc_actor
...

class Sink: public smoc_actor
...
\end{lstlisting}
\begin{itemize}
\item Using the configuration API requires to include the \lstinline!vpc_api.hpp! header.
\item Guarding configuration code using the preprocessor guard \lstinline!SYSTEMOC_ENABLE_VPC! is recommended and allows to compile the code in absence of the VPC library for functional simulation.
\end{itemize}
\end{frame}



%%%%%%%%%%%%%%%%%%%%%%%%%%%%%%%%%%%%%%%%%%%%%%%%%%%%%%%%%%%%%%%%%%%%%%%%%%%%%%
%%%%%%%%%%%%%%%%%%%%%%%%%%%%%%%%%%%%%%%%%%%%%%%%%%%%%%%%%%%%%%%%%%%%%%%%%%%%%%
\begin{frame}[fragile=singleslide]
\mode<presentation>{\frametitle{\insertsubsection\ -- Source Sink Example}}
\begin{lstlisting}
class NetworkGraph: public smoc_graph {
public:
  NetworkGraph(sc_module_name name) // network graph constructor
   : smoc_graph(name), source("src",this), sink("snk",this)
  {
    smoc_fifo<char> fifo("queue", 42);
    fifo.connect(source.out).connect(sink.in); // connect actors

#ifdef SYSTEMOC_ENABLE_VPC
...
#endif // SYSTEMOC_ENABLE_VPC
  }
...
\end{lstlisting}
\begin{itemize}
\item The network of actors is composed by a network graph.
\item Configuration code using the API mechanism is added to the network graph constructor.
\item Placing the code in the constructor is not mandatory, but eases access to the actor member variables (e.g., \lstinline!source!).
\end{itemize}
\end{frame}


%%%%%%%%%%%%%%%%%%%%%%%%%%%%%%%%%%%%%%%%%%%%%%%%%%%%%%%%%%%%%%%%%%%%%%%%%%%%%%
%%%%%%%%%%%%%%%%%%%%%%%%%%%%%%%%%%%%%%%%%%%%%%%%%%%%%%%%%%%%%%%%%%%%%%%%%%%%%%
\begin{frame}[fragile=singleslide]
\mode<presentation>{\frametitle{\insertsubsection\ -- Error Reporting}}
\begin{lstlisting}
#ifdef SYSTEMOC_ENABLE_VPC
  try {
    ...
  } catch (std::exception & e) {
    std::cerr << "Caught exception wile configuring VPC:\n" << e.what()
              << std::endl;
     exit(-1);
  }
#endif // SYSTEMOC_ENABLE_VPC
\end{lstlisting}
\begin{itemize}
\item The configuration API uses exceptions for error reporting.
\item Therefore, the configuration code shall be placed in a \lstinline!try{}! block.
\item Note the preprocessor guard \lstinline!SYSTEMOC_ENABLE_VPC! to enable compilation without VPC.
\end{itemize}
\end{frame}


%%%%%%%%%%%%%%%%%%%%%%%%%%%%%%%%%%%%%%%%%%%%%%%%%%%%%%%%%%%%%%%%%%%%%%%%%%%%%%
%%%%%%%%%%%%%%%%%%%%%%%%%%%%%%%%%%%%%%%%%%%%%%%%%%%%%%%%%%%%%%%%%%%%%%%%%%%%%%
\begin{frame}[fragile=singleslide]
\mode<presentation>{\frametitle{\insertsubsection\ -- Architecture Model}}
\begin{lstlisting}
    try {
      // convenience
      namespace VC = SystemC_VPC::Config;

      //create components
      VC::Component::Ptr cpu = VC::createComponent("CPU",
          VC::Scheduler::StaticPriority_NP);
      ...
\end{lstlisting}
\begin{itemize}
\item Here a namespace alias \lstinline!VC! is used to ease access the \lstinline!SystemC_VPC::Config! namespace.
\item A function \lstinline!VC::createComponent! is used to create a component.
\item Here a component using a non-preemptive static priority scheduling policy is modeled.
\item The \lstinline!VC::Component::Ptr CPA! is a handle (smart pointer) to the modeled component.
\end{itemize}
\end{frame}


%%%%%%%%%%%%%%%%%%%%%%%%%%%%%%%%%%%%%%%%%%%%%%%%%%%%%%%%%%%%%%%%%%%%%%%%%%%%%%
%%%%%%%%%%%%%%%%%%%%%%%%%%%%%%%%%%%%%%%%%%%%%%%%%%%%%%%%%%%%%%%%%%%%%%%%%%%%%%
\begin{frame}[fragile=singleslide]
\mode<presentation>{\frametitle{\insertsubsection\ -- Task Mapping}}
\begin{lstlisting}
      // mappings
      cpu->addTask(source);
      cpu->addTask(sink);

      // set priorities
      VC::setPriority(source, 0);
      VC::setPriority(sink, 1);
\end{lstlisting}
\begin{itemize}
\item The handle to the \lstinline!cpu! is used to register the mapping of actors.
\item Note: \lstinline!source! and \lstinline!src! are variables of type \lstinline!smoc_actor!.
\item Task priorities are configured using \lstinline!VC::setPriority!.
\end{itemize}
\end{frame}


%%%%%%%%%%%%%%%%%%%%%%%%%%%%%%%%%%%%%%%%%%%%%%%%%%%%%%%%%%%%%%%%%%%%%%%%%%%%%%
%%%%%%%%%%%%%%%%%%%%%%%%%%%%%%%%%%%%%%%%%%%%%%%%%%%%%%%%%%%%%%%%%%%%%%%%%%%%%%
\begin{frame}[fragile=singleslide]
\mode<presentation>{\frametitle{\insertsubsection\ -- Timing Provider}}
\begin{lstlisting}
      // timings
      VC::DefaultTimingsProvider::Ptr provider = cpu->getDefaultTimingsProvider();

      provider->add(VC::Timing("Source::source", sc_time(10, SC_NS),
          sc_time(10, SC_NS))); // dii, latency
      provider->add(VC::Timing("Sink::sink", sc_time(10, SC_NS))); // delay
      provider->add(VC::Timing("Source::source", sc_time(100, SC_NS)));
      provider->add(VC::Timing("Sink::sink", sc_time(100, SC_NS)));
\end{lstlisting}
\begin{itemize}
\item Execution times for actors are configured via a \lstinline!TimingsProvider!.
\item A default provider is used to configure individual timings for individual actions of actors.
\end{itemize}
\end{frame}


%%%%%%%%%%%%%%%%%%%%%%%%%%%%%%%%%%%%%%%%%%%%%%%%%%%%%%%%%%%%%%%%%%%%%%%%%%%%%%
%%%%%%%%%%%%%%%%%%%%%%%%%%%%%%%%%%%%%%%%%%%%%%%%%%%%%%%%%%%%%%%%%%%%%%%%%%%%%%
\begin{frame}[fragile=singleslide]
\mode<presentation>{\frametitle{\insertsubsection\ -- Default Route}}
\begin{lstlisting}
      // configure routing
      VC::ignoreMissingRoutes(true);
\end{lstlisting}
\begin{itemize}
\item So far the architecture and the task mapping is configured only.
\item The configuration of communication routes is still missing.
\item Here the default behavior is changed to ignore missing routes for a first simulation without modeled communication routes.
\end{itemize}
\end{frame}


%%%%%%%%%%%%%%%%%%%%%%%%%%%%%%%%%%%%%%%%%%%%%%%%%%%%%%%%%%%%%%%%%%%%%%%%%%%%%%
%%%%%%%%%%%%%%%%%%%%%%%%%%%%%%%%%%%%%%%%%%%%%%%%%%%%%%%%%%%%%%%%%%%%%%%%%%%%%%
\begin{frame}[fragile=singleslide]
\mode<presentation>{\frametitle{\insertsubsection\ -- Simulation Output}}
\begin{lstlisting}[language=xml]
             SystemC 2.2.0 --- Jan 16 2009 10:08:32
        Copyright (c) 1996-2006 by all Contributors
                    ALL RIGHTS RESERVED            
Note: VCD trace timescale unit is set by user to 1.000000e-09 sec.
top.Source> @ 0 s       send: 'H'                                 
top.Source> @ 100 ns    send: 'e'                                 
top.Sink> @ 100 ns      recv: 'H'                                 
...
SystemC: simulation stopped by user.                              
[VPC] overall simulated time: 3 us                                
\end{lstlisting}
\begin{itemize}
\item Simulation output from the example configured without routes.
\end{itemize}
\end{frame}


%%%%%%%%%%%%%%%%%%%%%%%%%%%%%%%%%%%%%%%%%%%%%%%%%%%%%%%%%%%%%%%%%%%%%%%%%%%%%%
%%%%%%%%%%%%%%%%%%%%%%%%%%%%%%%%%%%%%%%%%%%%%%%%%%%%%%%%%%%%%%%%%%%%%%%%%%%%%%
\begin{frame}[fragile=singleslide]
\mode<presentation>{\frametitle{\insertsubsection\ -- Transfer Timing}}
\begin{lstlisting}
      VC::ignoreMissingRoutes(true);

      VC::Component::Ptr bus = VC::createComponent("Bus");
      VC::Component::Ptr mem = VC::createComponent("Memory");

      VC::Timing transfer(sc_time(20,SC_NS), sc_time(20,SC_NS));
      bus->setTransferTiming(transfer);
      mem->setTransferTiming(transfer);
      cpu->setTransferTiming(transfer);
\end{lstlisting}
\begin{itemize}
\item Additional components are created to model a bus and a memory.
\item Components shall be configured with transfer timings.
\item A transfer timing models the time required to transfer a single token via a component (one hop).
\end{itemize}
\end{frame}


%%%%%%%%%%%%%%%%%%%%%%%%%%%%%%%%%%%%%%%%%%%%%%%%%%%%%%%%%%%%%%%%%%%%%%%%%%%%%%
%%%%%%%%%%%%%%%%%%%%%%%%%%%%%%%%%%%%%%%%%%%%%%%%%%%%%%%%%%%%%%%%%%%%%%%%%%%%%%
\begin{frame}[fragile=singleslide]
\mode<presentation>{\frametitle{\insertsubsection\ -- Multi-Hop Communication}}
\begin{lstlisting}
      VC::Route::Ptr writeRoute = VC::createRoute(getLeafPort(&source.out));
      VC::Route::Ptr readRoute = VC::createRoute(getLeafPort(&sink.in));
\end{lstlisting}
\begin{itemize}
\item \lstinline!VC::createRoute! is used to model a communication route for actor ports.
\item Note: \lstinline!geLeafPort! is used to access the actors port in case of hierarchically nesting  graphs.
\end{itemize}
\end{frame}


%%%%%%%%%%%%%%%%%%%%%%%%%%%%%%%%%%%%%%%%%%%%%%%%%%%%%%%%%%%%%%%%%%%%%%%%%%%%%%
%%%%%%%%%%%%%%%%%%%%%%%%%%%%%%%%%%%%%%%%%%%%%%%%%%%%%%%%%%%%%%%%%%%%%%%%%%%%%%
\begin{frame}[fragile=singleslide]
\mode<presentation>{\frametitle{\insertsubsection\ -- Mult-Hop Communication}}
\begin{lstlisting}
      writeRoute->addHop(cpu);
      writeRoute->addHop(bus).setPriority(0).setTransferTiming(
          VC::Timing(sc_time(10, SC_NS), sc_time(20, SC_NS)));
      writeRoute->addHop(mem);

      readRoute->addHop(mem);
      readRoute->addHop(bus);
      readRoute->addHop(cpu);
\end{lstlisting}
\begin{itemize}
\item A sequence of components is configured to represent multi-hop communication routes.
\item For simulation all delays occurring along a route plus additional scheduling and arbitration delays contribute to the overall communication delay.
\item Note: It is possible to configure communication timings and message priorities individually per hop in a route.
\end{itemize}
\end{frame}


%%%%%%%%%%%%%%%%%%%%%%%%%%%%%%%%%%%%%%%%%%%%%%%%%%%%%%%%%%%%%%%%%%%%%%%%%%%%%%
%%%%%%%%%%%%%%%%%%%%%%%%%%%%%%%%%%%%%%%%%%%%%%%%%%%%%%%%%%%%%%%%%%%%%%%%%%%%%%
\begin{frame}[fragile=singleslide]
\mode<presentation>{\frametitle{\insertsubsection\ -- Simulation Output}}
\begin{lstlisting}[language=xml]
             SystemC 2.2.0 --- Jan 16 2009 10:08:32
        Copyright (c) 1996-2006 by all Contributors
                    ALL RIGHTS RESERVED            
Note: VCD trace timescale unit is set by user to 1.000000e-09 sec.
Note: VCD trace timescale unit is set by user to 1.000000e-09 sec.
Note: VCD trace timescale unit is set by user to 1.000000e-09 sec.
top.Source> @ 0 s       send: 'H'
top.Source> @ 100 ns    send: 'e'
top.Sink> @ 150 ns      recv: 'H'
...
SystemC: simulation stopped by user.
[VPC] overall simulated time: 3720 ns
\end{lstlisting}
\begin{itemize}
\item Having a configured route in the examples causes different simulated times.
\end{itemize}
\end{frame}


%%%%%%%%%%%%%%%%%%%%%%%%%%%%%%%%%%%%%%%%%%%%%%%%%%%%%%%%%%%%%%%%%%%%%%%%%%%%%%
%%
%%%%%%%%%%%%%%%%%%%%%%%%%%%%%%%%%%%%%%%%%%%%%%%%%%%%%%%%%%%%%%%%%%%%%%%%%%%%%%
\subsection{API Reference}


%%%%%%%%%%%%%%%%%%%%%%%%%%%%%%%%%%%%%%%%%%%%%%%%%%%%%%%%%%%%%%%%%%%%%%%%%%%%%%
%%%%%%%%%%%%%%%%%%%%%%%%%%%%%%%%%%%%%%%%%%%%%%%%%%%%%%%%%%%%%%%%%%%%%%%%%%%%%%
\begin{frame}[fragile=singleslide]
\mode<presentation>{\frametitle{\insertsubsection\ -- Component}}
\begin{lstlisting}
Component::Ptr createComponent(std::string name,
    Scheduler::Type scheduler = Scheduler::FCFS);
\end{lstlisting}
\begin{itemize}
\item Create a component with default or explicitly given Scheduler.
\end{itemize}
\begin{lstlisting}
Component::Ptr getComponent(std::string name);
\end{lstlisting}
\begin{itemize}
\item Get an already created a component by name.
\end{itemize}
\begin{lstlisting}
bool hasComponent(std::string name);
\end{lstlisting}
\begin{itemize}
\item Check existence of a component by name.
\end{itemize}
\end{frame}


%%%%%%%%%%%%%%%%%%%%%%%%%%%%%%%%%%%%%%%%%%%%%%%%%%%%%%%%%%%%%%%%%%%%%%%%%%%%%%
%%%%%%%%%%%%%%%%%%%%%%%%%%%%%%%%%%%%%%%%%%%%%%%%%%%%%%%%%%%%%%%%%%%%%%%%%%%%%%
\begin{frame}[fragile=singleslide]
\mode<presentation>{\frametitle{\insertsubsection\ -- Component}}
\begin{lstlisting}
class Component {
public:
  void setScheduler(Scheduler::Type scheduler);
  Scheduler::Type getScheduler() const;
  void setTransferTiming(Timing transferTiming);
  Timing getTransferTiming() const;
  void addTask(ScheduledTask & actor);
  bool hasTask(ScheduledTask * actor) const;
  void setTimingsProvider(TimingsProvider::Ptr provider);
  TimingsProvider::Ptr getTimingsProvider();
  DefaultTimingsProvider::Ptr getDefaultTimingsProvider();
}
\end{lstlisting}
\begin{itemize}
\item The \lstinline!Component! class (\lstinline!Component::Ptr!) has various public member functions to get and set scheduler, transfer timings, mappings and timings provider.
\end{itemize}
\end{frame}



%%%%%%%%%%%%%%%%%%%%%%%%%%%%%%%%%%%%%%%%%%%%%%%%%%%%%%%%%%%%%%%%%%%%%%%%%%%%%%
%%%%%%%%%%%%%%%%%%%%%%%%%%%%%%%%%%%%%%%%%%%%%%%%%%%%%%%%%%%%%%%%%%%%%%%%%%%%%%
\begin{frame}[fragile=singleslide]
\mode<presentation>{\frametitle{\insertsubsection\ -- DefaultTimingsProvider}}
\begin{lstlisting}
class DefaultTimingsProvider {
  virtual void addDefaultActorTiming(std::string actorName, Timing timing);
  virtual void add(Timing timing);
}

\end{lstlisting}
\begin{itemize}
\item The easiest way to configure task execution times used the Component's \lstinline!DefaultTimingsProvider!
\item It allows to add timings to actors or to individual functions.
\end{itemize}
\end{frame}


%%%%%%%%%%%%%%%%%%%%%%%%%%%%%%%%%%%%%%%%%%%%%%%%%%%%%%%%%%%%%%%%%%%%%%%%%%%%%%
%%%%%%%%%%%%%%%%%%%%%%%%%%%%%%%%%%%%%%%%%%%%%%%%%%%%%%%%%%%%%%%%%%%%%%%%%%%%%%
\begin{frame}[fragile=singleslide]
\mode<presentation>{\frametitle{\insertsubsection\ -- Route}}
\begin{lstlisting}
Route::Ptr createRoute(const sc_port_base * leafPtr,
    Route::Type type = Route::StaticRoute);
\end{lstlisting}
\begin{itemize}
\item Create a route using a ``leaf'' port of an actor.
\end{itemize}
\begin{lstlisting}
class Route {
public:
  bool getTracing() const;
  void setTracing(bool tracing_);
  Hop & addHop(Component::Ptr component);
  void addTiming(Component::Ptr hop, Timing);
}
\end{lstlisting}
\begin{itemize}
\item The \lstinline!Route! class has public member functions to add hops, configure individual timings, and enable message tracing.
\end{itemize}
\end{frame}


%%%%%%%%%%%%%%%%%%%%%%%%%%%%%%%%%%%%%%%%%%%%%%%%%%%%%%%%%%%%%%%%%%%%%%%%%%%%%%
%%%%%%%%%%%%%%%%%%%%%%%%%%%%%%%%%%%%%%%%%%%%%%%%%%%%%%%%%%%%%%%%%%%%%%%%%%%%%%
\begin{frame}[fragile=singleslide]
\mode<presentation>{\frametitle{\insertsubsection\ -- Hops}}
\begin{lstlisting}
class Hop {
public:
  Hop & setPriority(size_t priority_);
  Hop & setTransferTiming(Timing transferTiming_);
  Component::Ptr getComponent() const;
  size_t getPriority() const;
  Timing getTransferTiming() const;
\end{lstlisting}
\begin{itemize}
\item The \lstinline!Hop! class has public member to get and set individual message priorities and timings.
\item Furthermore getting the referenced \lstinline!Component! is possible.
\end{itemize}
\end{frame}


%%%%%%%%%%%%%%%%%%%%%%%%%%%%%%%%%%%%%%%%%%%%%%%%%%%%%%%%%%%%%%%%%%%%%%%%%%%%%%
%%%%%%%%%%%%%%%%%%%%%%%%%%%%%%%%%%%%%%%%%%%%%%%%%%%%%%%%%%%%%%%%%%%%%%%%%%%%%%
\begin{frame}[fragile=singleslide]
\mode<presentation>{\frametitle{\insertsubsection\ -- Configuration}}
\begin{lstlisting}
void setPriority(ScheduledTask & actor, size_t priority);
\end{lstlisting}
\begin{itemize}
\item The API allows to set task priorities.
\end{itemize}
\begin{lstlisting}
void ignoreMissingRoutes(bool ignore);
\end{lstlisting}
\begin{itemize}
\item If required, not configured (missing) route are ignored for simulation.
\end{itemize}
\end{frame}



