%%%%%%%%%%%%%%%%%%%%%%%%%%%%%%%%%%%%%%%%%%%%%%%%%%%%%%%%%%%%%%%%%%%%%%%%%%%%%%
%%%%%%%%%%%%%%%%%%%%%%%%%%%%%%%%%%%%%%%%%%%%%%%%%%%%%%%%%%%%%%%%%%%%%%%%%%%%%%
\begin{frame}[t]
\mode<presentation>{\frametitle{\insertsubsection\ }}
\begin{itemize}
\item \SystemCoDesigner{} uses \SysteMoC{} and \VPC\ for functional and performance simulation
\end{itemize}
\begin{itemize}
\item \SysteMoC{} allows for functional modeling and simulation
\end{itemize}

\begin{itemize}
\item \VPC\ (VPC) allow for performance modeling and simulation
\end{itemize}

\begin{itemize}
\item Together, \SysteMoC{} and \VPC\, allow for combined functional and performance simulation
\end{itemize}

\begin{itemize}
\item Both, \SysteMoC{} and \VPC\, are implemented as individual libraries on top of SystemC
\end{itemize}

\end{frame}


%%%%%%%%%%%%%%%%%%%%%%%%%%%%%%%%%%%%%%%%%%%%%%%%%%%%%%%%%%%%%%%%%%%%%%%%%%%%%%
%%%%%%%%%%%%%%%%%%%%%%%%%%%%%%%%%%%%%%%%%%%%%%%%%%%%%%%%%%%%%%%%%%%%%%%%%%%%%%
\begin{frame}[t]
\mode<presentation>{\frametitle{\insertsection\ -- VPC }}
\begin{itemize}
\item A \VPC\ ...
\item ... is a SystemC module
\item ... models task execution times
\item ... models resource contention, scheduling, and arbitration
\end{itemize}

\begin{itemize}
\item A \VPC\ ...
\item ... is not an Instruction Set Simulator
\item ... is not Software
\item ... is not Hardware
\end{itemize}

\begin{itemize}
\item But, a \VPC\ ...
\item ... allows to model a Processor or a HW component
\item ... allows to model a communication resource
\end{itemize}

\end{frame}



%%%%%%%%%%%%%%%%%%%%%%%%%%%%%%%%%%%%%%%%%%%%%%%%%%%%%%%%%%%%%%%%%%%%%%%%%%%%%%
%%%%%%%%%%%%%%%%%%%%%%%%%%%%%%%%%%%%%%%%%%%%%%%%%%%%%%%%%%%%%%%%%%%%%%%%%%%%%%
\begin{frame}[t]
\mode<presentation>{\frametitle{\insertsection\ -- \SysteMoC\ }}

\begin{figure}
\centering
\resizebox{0.7\columnwidth}{!}{\input{vpc-systemoc-fig.tex}}
\end{figure}

\begin{itemize}
\item A functional model in \SysteMoC\ is given by ...
\item ... a set of actors (e.g., Source and Sink)
\item ... a set of communication queues connecting actors 
\end{itemize}
\end{frame}



%%%%%%%%%%%%%%%%%%%%%%%%%%%%%%%%%%%%%%%%%%%%%%%%%%%%%%%%%%%%%%%%%%%%%%%%%%%%%%
%%%%%%%%%%%%%%%%%%%%%%%%%%%%%%%%%%%%%%%%%%%%%%%%%%%%%%%%%%%%%%%%%%%%%%%%%%%%%%
\begin{frame}[t]
\mode<presentation>{\frametitle{\VPC\ }}

\begin{figure}
\centering
\resizebox{0.7\columnwidth}{!}{\input{vpc-systemoc-fig.tex}}
\end{figure}

\begin{itemize}
\item An architecture model in VPC is given by ...
\item ... a set of components (e.g., CPU, Bus, and Mem )
\item ... a mapping of actors and queues to the set of components
\end{itemize}
\end{frame}









