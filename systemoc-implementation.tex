\section{Implementation of the firing FSM}\label{sec:systemoc-implementation}

In contrast to other approaches~\cite{herrerasystemc:2004, PS:2005, PS:2004} we model our firing FSM with \emph{expression templates} \cite{veldhuizen:1995}.
More traditional approaches would either build the abstract syntax tree directly from the activation pattern, via operator overloading, or would use callback functions to implement activation patterns.
In the first case a costly interpretation phase of the abstract syntax tree is necessary to evaluate activation patterns.
In the second case a parser would be necessary to extract the abstract syntax tree from the source code.
Therefore, the information encoded in the firing FSM would not be available for the SystemC elaboration phase.
As an example we pick the activation pattern on transition $t_2$ of the \code{SqrLoop} actor $a_2$, as shown in Figure~\ref{fig:ast-t2-sqrloop}.
% The activation patterns used to describe the firing FSM of an actor as seen in \ref{ex:systemoc-sqrloop-fsm-def} on page \pageref{ex:systemoc-sqrloop-fsm-def} and defined in \ref{syn:systemoc-fsm-bnf} and \ref{syn:systemoc-fsm-bnf-2} are encoded in C++ as so called \emph{expression templates} \cite{veldhuizen:1995}.
The constructed expression template for an activation pattern is a tree of nest template types which corresponds to the \emph{abstract syntax tree} of the activation pattern.%, as shown in Figure~\ref{fig:ast-t2-sqrloop}.

%\begin{figure}[h]
%\centering
%\resizebox{\columnwidth}{!}{\input{ast-fig.tex}}
%%\input{ast-fig.tex}
%%\includegraphics[width = 5in]{actor-sqrloop}
%\caption{\label{fig:ast-t2-sqrloop}%
%Visual representation of the \emph{abstract syntax tree} and its corresponding \emph{expression template} derived from the activation pattern used in transition $t_2$ of the \code{SqrLoop} actor $a_2$ from Figure~\ref{fig:actor-sqrloop} on page \pageref{fig:actor-sqrloop}.
%}
%\end{figure}

Using expression templates allows us to use both compile time code transformation and to derive at C++ compile time the \emph{abstract syntax tree} for our activation patterns enabling:
(i) extraction of the FIFO channels used in an activation pattern to generate sensitivity lists,
(ii) compile time code generation for parts of an activation pattern only dependent on the actor state, e.g., as seen in Figure~\ref{fig:ast-compile-time-transform}, or
(iii) generation of a XML representation of the firing FSM, e.g., as seen in Figure~\ref{fig:xml-t2-sqrloop}, for later usage in the design flow.

\begin{figure}[h]
\centering
\resizebox{\columnwidth}{!}{\input{ast-compile-time-transform-fig.tex}}
%\input{ast-fig.tex}
%\includegraphics[width = 5in]{actor-sqrloop}
\caption{\label{fig:ast-t2-sqrloop}\label{fig:ast-compile-time-transform}%
Compile time code transformation of an activation pattern into a sensitivity list used for scheduling and an functionality state dependent part used to check transition readiness after the scheduling step.
}
\end{figure}

The sensitivity AST is only evaluated after a change in the number of available or free tokens in the monitored FIFO channels.
The actor state dependent AST parts only after its corresponding sensitivity AST evaluated to \code{true}.
Note that in general arbitrarily complex parts dependent on the actor state of an activation pattern can be identified.
For these actor state dependent AST parts dedicated code is generated at C++ compile time to evaluate them.

\begin{figure}[h]
\begin{verbatim}
<transition nextstate="id9" action="SqrLoop::copyApprox">
  <ASTNodeBinOp valueType="b" opType="DOpBinLAnd">
    <lhs><ASTNodeBinOp valueType="b" opType="DOpBinLAnd">
      <lhs><ASTNodeBinOp valueType="b" opType="DOpBinGe">
        <lhs><PortTokens valueType="j" portid="id6"/></lhs>
        <rhs><Literal valueType="j" value="1"/></rhs>
      </ASTNodeBinOp></lhs>
      <rhs><MemGuard valueType="b" name="SqrLoop::check"></rhs>
    </ASTNodeBinOp></lhs>
    <rhs><ASTNodeBinOp valueType="b" opType="DOpBinGe">
      <lhs><PortTokens valueType="j" portid="id8"/></lhs>
      <rhs><Literal valueType="j" value="1"/></rhs>
    </ASTNodeBinOp></rhs>
  </ASTNodeBinOp>
</transition>
\end{verbatim}
\caption{\label{fig:xml-t2-sqrloop}%
XML representation of the transition $t_2$ of the \code{SqrLoop} actor $a_2$ including the \emph{abstract syntax tree} derived from the activation pattern used in the transition.
}
\end{figure}




% LocalWords: SysteMoC
