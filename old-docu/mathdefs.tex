\chapter{Mathdefs}
% Before we can review the tagged signal model some mathematical notations are needed.

We use the `$.$'-operator, e.g., $u.x$, for member access, e.g., $x$, of tuples whose members have been explicitly named in their definition, e.g., $u = (x, y) \in U \subseteq \mathbf{N}^2$.
Furthermore this member access operator has a trivial pointwise extenstion to sets of tuples, e.g., $U.x = \bigcup{}_{u \in U} u.x$.

Let $\{x_1,x_2\}$ denote a set and $[y_1,y_2,y_3,\ldots]$ denote a set
with total order $y_1 \le y_2 \le y_3 \le \ldots$, e.g., $\{x_1,x_2\} = \{x_2,x_1\}$ but
$[y_1,y_2] \ne [y_2,y_1]$ and $\{x_1,x_1\} = \{x_1\}$ but $[ y_1, y_2, y_1, y_3]$
is a \emph{illformed totally ordered set} because it does not define a total order,
as $y_1 \le y_2 \wedge y_2 \le y_1$ but $y_1 \ne y_2$.

Let $\mathbb{N} = [1,2,3,\ldots]$ denote the set of natural numbers,
$\mathbb{N}_\infty = [1,2,3,\ldots\infty]$ the set of natural numbers including infinity,
$\mathbb{N}_n = [1,2,3,\ldots,n] \subseteq \mathbb{N}_\infty$ the counting set from one to $n$ (Note that $\mathbb{N}_0 = \emptyset$),
$\mathbb{Z} = [\ldots,-2,-1,0,1,2,\ldots]$ denote the set of integers,
%$\mathbb{Z}^-_0 = [\ldots,-3,-2,-1,0]$ the set of nonpositive integers,
$\mathbb{Z}^+_0 = [0,1,2,3,\ldots]$ the set of nonnegative integers,
$\mathbb{Z}^{+\infty}_0 = [0,1,2,3,\ldots\infty]$ the set of nonnegative integers including infinity.
%$\mathbb{Z}^- = [\ldots,-3,-2,-1]$ the set of negative integers.

A totally ordered set $X$ is called \emph{two-sided discrete}
if it is \emph{order isomorphic} to a subset of the integers $Z_X \subseteq \mathbb{Z}$,
%and let $\Sigma(X): Z_X \to X$ denote this order isomorphism,
i.e., $\exists{\Sigma(X): Z_X \to X, \Sigma(X)\textrm{ is a \emph{bijection}}}:
 \forall{i,j \in Z_X}: i \le j \iff \Sigma(X)(i) \le \Sigma(X)(j)$.
Intuitively this means, any two elements
$x_1, x_2 \in X$ have only a finite number of other elements between them.

A totally ordered set $X$ is called \emph{one-sided discrete} or \emph{discrete}
if it is \emph{order isomorphic} to a subset of the natural numbers $\mathbb{N}_{|X|}$
and let $\Sigma(X): \mathbb{N}_{|X|} \to X$ denote this order isomorphism,
i.e., $\exists{\Sigma(X): \mathbb{N}_{|X|} \to X, \Sigma(X)\textrm{ is a \emph{bijection}}}:
 \forall{i,j \in \mathbb{N}_{|X|}}: i \le j \iff \Sigma(X)(i) \le \Sigma(X)(j)$.
Intuitively this means,
any two elements $x_1, x_2 \in X$ have only a finite number of other elements
between them and there exists a least element $x_{least}, \forall{x \in X}: x_{least} \le x$.
For example, given the discrete set $X = [$`A', `B', `C'$]$ then
$\Sigma(X)(1) =$ `A',  $\Sigma(X)(2) =$ `B', and $\Sigma(X)(3) =$ `C'.

Let $X^* = \bigcup_{n \in \mathbb{Z}^+_0} X^n$ denote the set of all
\emph{tuples} of $X$ also called \emph{finite sequences} of $X$ and
$X^{**} = \bigcup_{n \in \mathbb{Z}^{+\infty}_0} X^n$ denote the set of all
finite and infinite \emph{sequences} of $X$. A sequence
$\mathbf{x} = (x_1,x_2,\ldots,x_n) \in X^n, n \in \mathbb{Z}^{+\infty}_0$
can also be thought of as a function $\mathbf{x} : \mathbb{N}_n \to X$.
For example, given the sequence $X = ($`A', `B', `A', `C'$)$ then
$X(1) =$ `A',  $X(2) =$ `B', $X(3) =$ `A', and $X(4) =$ `C'.
Note that a function is also a relation and a relation is also a set, e.g
$\mathbf{x} : \mathbb{N}_n \to X \equiv \{ (i,x_i) \mid x_i = \mathbf{x}(i), i \in \mathbb{N}_n \}$.
Note also that the $\Sigma$ function converts a discrete set to a sequence,
e.g., $\Sigma([1,2,3]) = (1,2,3)$.

A set with an associated partial order is called a
\emph{poset}. Let $\sqsubseteq$ called \emph{prefix order} denote the
associated partial order on the set $V^{**}$ of all finite and infinite sequences of $V$.
And let $\mathbf{v}_1 \sqsubseteq \mathbf{v}_2$ denote that
$\mathbf{v}_1$ is a prefix of the sequence $\mathbf{v}_2$, i.e.,
$\mathbf{v}_1 \sqsubseteq \mathbf{v}_2 \iff \forall{n \in T(\mathbf{v}_1)}: \mathbf{v}_1(n) = \mathbf{v}_2(n)$.

An \emph{upper bound} of a subset $X \subseteq V^{**}$ is an element
$\mathbf{v} \in V^{**}$ where every sequence $\mathbf{x}$ in $X$ is a
prefix of $\mathbf{v}$, i.e., $\forall{\mathbf{x} \in X}: \mathbf{x} \sqsubseteq \mathbf{v}$.
A \emph{least upper bound} of a subset $X \subseteq V^{**}$, written $\sqcup X$,
is an upper bound which is a prefix of every other upper bound of $X$.
A \emph{lower bound} and \emph{greatest lower bound} of $X \subseteq V^{**}$,
written $\sqcap X$, are defined respectively.

A \emph{chain} in a poset $X$ is a totally ordered subset of $X$. A
\emph{bottom element} of a poset $X$, if it exists, is an element of the
poset which is also a greatest lower bound of the whole poset $\sqcap X \in X$.
A \emph{complete partial order} is a poset $X$ with a bottom element
in which every chain $Y \subseteq X$ has a least upper bound
$\sqcup Y$. The set $V^{**}$ of all finite and infinite sequences of $V$
is a complete partial order with bottom element $\lambda$, the
empty sequence.

This definitions can be trivially generalized to tupples of sequences
$\upsilon \in (V^{**})^N$, i.e., $\upsilon_1 \sqsubseteq \upsilon_2 \iff
\forall{n \in \mathbb{N}_N}: \upsilon_1(n) \sqsubseteq \upsilon_2(n)$.
With this generalization the set $(V^{**})^N$ of all $N$-tupples of finite and
infinite sequences of $V$ is also a complete partial order with
bottom element $\Lambda = (\lambda, \lambda, \ldots, \lambda)$, the $N$-tupple
of empty sequences $\lambda$.

A function $F: (V^{**})^N \to (V^{**})^M$ is \emph{monotonic} if
adding additional elements to the input sequence tupple $\upsilon_{in} \in (V^{**})^N$
results in additional elements on the output sequence tupple
$\upsilon_{out} = F(\upsilon_{in}) \in (V^{**})^M$,i.e.,
$\forall{\upsilon_{in_1}, \upsilon_{in_2} \in (V^{**})^N}:
\upsilon_{in_1} \sqsubseteq \upsilon_{in_2} \implies F(\upsilon_{in_1}) \sqsubseteq F(\upsilon_{in_2})$.
This can be considered as a untimed notion of causality
where the additional elements to the input sequence correspond
to input tokens and the additional elements on the output sequence
are the tokens produced from the consumption of these input tokens.

A function $F: (V^{**})^N \to (V^{**})^M$ is \emph{continuous} if
for every chain $Y$ in $(V^{**})^N$, $F(Y)$ has a least upper bound
$\sqcup F(Y)$ which is equal to the function $F$ applied to the
upper bound $\sqcup Y$ of the chain $Y$,
i.e., $\forall{Y \subseteq (V^{**})^N, Y\textrm{ is a chain}}: F(\sqcup Y) = \sqcup F(Y)$.
Where $F : 2^{\left( (V^{**})^N \right)} \to 2^{\left( (V^{**})^M \right)}$
is the pointwise extension of $F : (V^{**})^N \to (V^{**})^M$, i.e.,
$F(Y) = \{ F(\upsilon) \mid \upsilon \in Y\}$. Note that
every continuous function is also monotonic but the reverse
is not necessarily true.


