\chapter{\SysteMoC{} - XML formats}\label{sec:systemoc-ngx}

\SysteMoC{} enables an XML representation of the network graph as defined in Section~\ref{sec:network-graph} to be automatically extracted, e.g., a skeleton of the XML representation of the network graph shown in Figure~\ref{fig:ng-sqrroot} can be seen in the following example.

%\verbatiminput{sqrroot.ngx}

\begin{example}
Skeleton of a network graph XML representation:
\begin{verbatim}
<?xml version="1.0"?>
<!DOCTYPE networkgraph SYSTEM "networkgraph.dtd">
<networkgraph name="sqrroot">
  <!-- All actors and channels of a network graph are given below.
       Where both actors and channels are represented via the XML
       process tag.
    -->
  <!-- Process tag describing SqrRoot actor a1 -->
  <process name="sqrroot.a1" type="actor" id="id1">
    <port name="sqrroot.a1.smoc_port_out_0" type="out" id="id2"/>
    <!-- Contains more tags describing the actor in more detail -->
    ...
  </process>
  ...
  <!-- Process tag describing FIFO channel c1 -->
  <process name="sqrroot.smoc_fifo_0" type="fifo" id="id24">
    <port name="sqrroot.smoc_fifo_0.in"  type="in"  id="id25"/>
    <port name="sqrroot.smoc_fifo_0.out" type="out" id="id26"/>
    <!-- Contains more tags describing the FIFO in more detail -->
    ...
  </process>
  ...
  <!-- Process tag describing SqrLoop actor a2 -->
  <process name="sqrroot.a2" type="actor" id="id4">
    <port name="sqrroot.a2.smoc_port_in_0"  type="in"  id="id5"/>
    <port name="sqrroot.a2.smoc_port_in_1"  type="in"  id="id6"/>
    <port name="sqrroot.a2.smoc_port_out_0" type="out" id="id7"/>
    <port name="sqrroot.a2.smoc_port_out_1" type="out" id="id8"/>
    <!-- Contains more tags describing the actor in more detail -->
    ...
  </process>
  ...
  <!-- Edge tags describing the FIFO connection from a1.o1 -> a2.i1 -->
  <edge name="sqrroot.smoc_fifo_0.to-edge"   source="id2"  target="id25"/>
  <edge name="sqrroot.smoc_fifo_0.from-edge" source="id26" target="id5"/>
  ...
</networkgraph>
\end{verbatim}
\end{example}

The actors and channels of a network graph are represented as \code{process} tags contained in the \code{networkgraph} tag.
The \code{process} tags are marked via the attribute \code{type} as representing an actor or FIFO channel.
The edges of the network graph are represented as \code{edge} tags.

\begin{example}\label{ex:xml-t2-sqrloop}%
XML representation of the transition $t_2$ of the \code{SqrLoop} actor $a_2$ including the \emph{abstract syntax tree} derived from the activation pattern used in the transition.
\begin{verbatim}
<transition nextstate="id9" action="SqrLoop::copyApprox">
  <ASTNodeBinOp valueType="b" opType="DOpBinLAnd">
    <lhs><ASTNodeBinOp valueType="b" opType="DOpBinLAnd">
      <lhs><ASTNodeBinOp valueType="b" opType="DOpBinGe">
        <lhs><PortTokens valueType="j" portid="id6"/></lhs>
        <rhs><Literal valueType="j" value="1"/></rhs>
      </ASTNodeBinOp></lhs>
      <rhs><MemGuard valueType="b" name="SqrLoop::check"></rhs>
    </ASTNodeBinOp></lhs>
    <rhs><ASTNodeBinOp valueType="b" opType="DOpBinGe">
      <lhs><PortTokens valueType="j" portid="id8"/></lhs>
      <rhs><Literal valueType="j" value="1"/></rhs>
    </ASTNodeBinOp></rhs>
  </ASTNodeBinOp>
</transition>
\end{verbatim}
\end{example}
