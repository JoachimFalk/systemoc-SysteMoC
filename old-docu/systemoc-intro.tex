\chapter{Introduction}\label{sec:intro}

%\vspace{-8mm}
%\begin{center}\parbox{12cm}
%{\itshape
Actor-based design is based on composing a system of communicating processes called \emph{actors}, which can only communicate with each other via channels.
However, \emph{actor-based design} does not constrain the communication behavior of its actors therefore making analyses of the system in general impossible.
In a \emph{model-based design} methodology the underlying \emph{Model of Computation} (MoC) is known additionally which is given by a predefined type of communication behavior and a scheduling strategy for the actors.
% Thus, it constraints the communication infrastructure of concurrent system to message passing over a predefined network graph of channels enabling XXX.
%They are comparable to design patterns as used in software engineering and the basis for model-based design. %, which is a methodology for creating executable specifications which only use predefined patterns for communication which correspond to different MoCs.
%Moreover, MoCs represent well known trade-offs between expressiveness and analyzability.
We present a library based on the design language SystemC called \SysteMoC{} which provides a simulation environment for model-based designs.
The library-based approach unites the advantage of executability with analyzability of many expressive MoCs.
%The base MoC of our library is a generalization of \emph{FunState} (Functions driven by State machines).
%We will introduce the syntax and semantics supported by \SysteMoC{} as well as discuss the simulation environment and present first results of using \SysteMoC{} for modeling and simulation of signal processing applications.
% \SysteMoC{} defines a coding style for specifying the communication behavior of actors as finite state machines and the communication infrastructure as a network graph of communicating actors, as well as functionality to extract this information from the executable specification.
% Therefore we provide the basis for model-based design in SystemC enabling automatic extraction and classification of the communication behavior of actors.
% Thus later enabling us to use analysis for optimized code generation as well as efficient system property checks by automatic verification tools.
%Finally, we compare the simulative performance of \SysteMoC{} with other executable languages such as C++, regular SystemC, and modelling environments such as Ptolemy II.
%}
%\end{center}

% The basic tenet of this
% methodology is to move away from manual coding from informal
% specifications by capturing embedded software functional
% and non-functional requirements from abstract mathematical
% models. Clearly, a mathematical model offers a
% common ground for a systematic and coherent integration
% of diverse efforts in system specification, design, synthesis
% (code generation), analysis (validation), execution (runtime
% support), and maintenance (design evolution).

Due to rising design complexity, it is necessary to increase the level of abstraction at which systems are designed.
This can be achieved by model-based design which makes extensive use of so-called \emph{Models of Computation} \cite{embsft:2002} (MoCs).
MoCs are comparable to design patterns known from the area of software design \cite{gamma:1995}.
On the other hand, industrial embedded system design is still based on design languages like C, C++, Java, VHDL, SystemC, and SystemVerilog which allow unstructured communication.
Even worse, nearly all design languages are Turing complete making analyses in general impossible.
This precludes the automatic identification of communication patterns out of the many forms of interactions, e.g., shared variables and various ways of message passing between processes.
%To make industry benefit from the best of both worlds, engineers must restrict themselves to use certain coding styles and subsets of a design language.
%This results in a model-based design methodology that permits automatic analysis, identification, and extraction of MoCs at the source code level.

%\begin{figure}[t]
%\centering
%\resizebox{\columnwidth}{!}{\input{SysteMoC-Goals-fig.tex}}
%\caption{\label{fig:SysteMoC-Goals}The Goal of the \SysteMoC{}-framework
%  is the identification and extraction of models of computation
%  from a subset of SystemC. Later the extracted MoC can be used
%  in a model-based design methodology, e.g., in design space exploration
%  and for code generation.
%}
%\end{figure}

% The overall methodology is depicted in Figure~\ref{fig:SysteMoC-Goals}.

%In this paper, we propose the \SysteMoC{} approach.
%The basic MoC of the \SysteMoC{}-library is \emph{FunState} (Functions driven by State machines) \cite{stgzet:2001}.
%FunState models express their communication behavior by \emph{Finite State Machines} (FSM).
%Analyzing these FSMs together with the topology of a given SystemC design permits the extraction and analysis of the underlying MoC to the given design.
%This is a prerequisite for later optimization or even for design automation approaches.
%In this paper, we will focus on the \SysteMoC{} syntax and semantics as well as the simulation environment for \SysteMoC{} designs.
%
%The rest of this paper is structured as follows:
%In Section~\ref{sec:related-work}, we discuss related work.
%In Section~\ref{sec:systemoc-syntax}, we present \SysteMoC{} syntax and semantics.
%In Section~\ref{sec:systemoc-implementation}, details pertaining to our implementation of the \SysteMoC{} simulation environment are presented.
%In Section~\ref{sec:systemoc-casestudy}, we compare the \SysteMoC{} simulation performance with other approaches,
%and we conclude the present paper in Section~\ref{sec:conclusions}.
%We will use the example of an approximating square root algorithm throughout the paper to illustrate our approach.

%\section{Basic concepts}
 
Instead of a monolithic approach for representing an executable specification of an embedded system as done using many design languages, we will use a refinement of \emph{actor-oriented} design.
In actor-oriented design, \emph{actors} only communicate with each other via \emph{channels} instead of method calls as known
in object-oriented design.
\SysteMoC{}~\cite{fht:2006} is a library based on SystemC that allows to describe and simulate communicating actors, which are divided into their \emph{actor functionality} and their \emph{communication behavior} encoded as an explicit finite state machine.
In the following, the syntax and semantics of \SysteMoC{} designs is discussed.

Using \SysteMoC, many important models of computation may be described such as 
SDF \cite{Lee87b:1987}, CSDF \cite{belp:1996, eblp:1994}, Boolean Dataflow, 
Kahn Process Networks \cite{Kahn:1974}, 
and many process networks, also communicating sequential processes (CSP)
\cite{csphoare:1985}, and many others \cite{tszet:1999, LeeDenotialDF:1997,
embsft:2002, Lee98, Eker, Tei97}.

A \SysteMoC{} actor contains 3 basic elements:

\begin{itemize}
\item {\em Network graph}: Each application is modeled 
by a network graph of communicating actors.
\item {\em Actor classes}: Each actor is defined 
by a class that contains several 
so-called {\em actions} that are basic functional blocks implemented as 
C++ methods that do computations on tokens on inputs and output ports 
and the internal state of an actor. The actor itself thus possesses an 
implementation in the form of a C++ class, and
each actor communicates with other actors through a certain number 
of input ports and a certain number of output ports. 
The basic entity of data transfer is regulated by the notion of 
{\em tokens}. Apart from the existence or absence of tokens, an actor 
may check and do computations also on values of these tokens. 
In this paper, we do not make any assumptions on the types 
of tokens exchanged between actors. 
\item {\em Firing finite state machine (FSM)}: The behavior of each actor is ruled 
by an explicit finite state machine that checks conditions on the input ports, 
output ports and internal state. If a certain firing rule is satisfied, 
a state transition will be taken. During the state transition, an {\em action}
is called. Once this action has finished execution, the new state is 
taken. The complete state of an actor is described by this explicit 
state of the actor and the state as given by its (local) member variables.
Depending on the model of computation, 
the actor ports may be connected to different types of so-called {\em channels} such 
as channels with FIFO semantics, or rendez-vous channels.
\end{itemize}

% LocalWords: SysteMoC
