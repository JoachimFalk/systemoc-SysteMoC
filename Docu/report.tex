\newcommand{\resdir}{../hscd-tex-res/}

\input{\resdir format}

\begin{document}
\hscdtitle{Representing MoCs in SystemC}{Joachim Falk}{\today}
\clearpage
\tableofcontents
\clearpage

\section{Introduction\label{intro}}
Due to rising design complexity it is necessary to increase
the level of abstraction at which systems are designed. Using
Models of Computation in system design can achieve this goal.
Some of the benefits of MoCs are:
\begin{enumerate}
\item MoCs can be used to mathematically reason about communication
patterns.

\item Knowledge about MoCs can be used by code generators to produce
optimized code.

\item Knowledge about MoCs can be used by verification tools to
check system properties.
\end{enumerate}
However for these benefits to be practically realized the
MoC must be extractable from the source code. Because the general case
is impossible to solve for a turing complete programming language
\footnote{Halting problem} a machine analyzable coding subset for
MoCs has to be defined.

Due to it's high abstraction level and possibilities for both
hardware and software refinements we have chosen SystemC as
our language for system design. In the following we propose a
coding subset of SystemC which faciliates automatic extraction
of the MoC of a SystemC design.

Instead of a Monolitic approach to represent an executable specification
in SystemC we choose too represent the specification
by combining severel independent concepts.

\begin{enumerate}
\item Node functionality

The specification is devided into several communicating actors.
The actor functionality is the algorithmic description
which is responsible for transforming data values. The firering behaviour
of the actor must not be altered in the node functionality.

\item Node interface

The node interface are the set of operators which are used to
define the firering behaviour of the actor.

\item Network graph

The actors have ports which are their only means of communication.
These ports are either input ports or output ports.
The network graph is a directed graph which specifies the connections from
output ports to input ports off the actors.
But no semantic is attached to these connections
only the information that the connections exist.

\item
Channel type

The channel type defines the semantic for the connections
the network graph specifies. Not per connection but for all
connections in the network graph uniformly.
\end{enumerate}

The node functionality is not


\clearpage
\section*{Dust bin}


to a new higher level hardware description
language for both system specification and system implementation is
desireable. One representative of this kind of hardware description
language is SystemC. Another challenge imposed by rising design complexity
is the exploding design space, which can no longer be reasonably explored
by humans.


If the Model of Computation is restricted to Dataflow the
MODES tool for automated Design Space exploration can be used. The MODES
tool uses a Graph model as an abstraction. The Graph model is called the
Specification Graph (following known as $G_{S}(V_{S},E_{S})$) which has as
induced Subgraphs a Problem Grap (following known as $G_{P}(V_{P},E_{P}) \subset G_{S}$) and
an Architecture Graph (following known as $G_{A}(V_{A},E_{A}) \subset G_{S}$).
... Say more Stuff here !!!
\par
To include the MODES tool into the Design Flow an automatic mapping between
the Graph model of MODES and the real world Design in a Hardware Description
Language must be implemented. The presented tool is concerned with the automated
mapping of a Design in SystemC, which uses the Dataflow Model of Computation,
to the MODES PG. Clearly not the full semantic of SystemC can be mapped to the
PG, only Dataflow is supported. And multiple ways exists to implement
Dataflow in SystemC. Therefore a subset of SystemC was choosen which not only
is restricted to Dataflow but to a certain way to implement Dataflow in SystemC.
\par
Following things have to be mapped from SystemC to the MODES PG:
\begin{enumerate}
\item Dataflow Nodes

These are mapped to SC\_MODULES in SystemC. They can have any semantic internally
but must export Dataflow semantic externally. These modules have only sc\_fifo\_in/sc\_fifo\_out
ports for communication. The only exception is an optional sc\_in\_clk for clock distribution.

Example:
\begin{verbatim}
SC_MODULE(design_mul2) {
 public:
  sc_in_clk                   clk;
  sc_fifo_in< tokentype_ty >  in;
  sc_fifo_out< tokentype_ty > out;

 private:
  void worker_thread( void ) {
    while ( 1 ) {
      out.write( in.read() * 2 );
    }
  }

 public:
  SC_CTOR(design_mul2) {
    SC_THREAD(worker_thread);
  }
};
\end{verbatim}
\item Datadependencies
\end{enumerate}

\par


\par
You can use labels as in Section~\ref{intro}.

\section{About Your Work}
You can place figures in your text as shown in Figure~\ref{yourfigure}.
\begin{figure}
\centering
\includegraphics[width=4in]{\resdir codesign.pdf}
\caption{\label{yourfigure}Your Figure}
\end{figure}
\par
You can cite any literatur by using the $\backslash$cite command. 
Here, an example: \cite{btt:1998}.
\subsection{First ideas}
\subsection{Obstacles}
\subsection{Results}

\clearpage
\appendix
\bibliographystyle{alpha}
\bibliography{literature}

\end{document}
