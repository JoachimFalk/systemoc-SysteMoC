\section{SysteMoc Idea}

{\bf FIXME: Translate into English}

Nahezu alle Anwendungen lassen sich auf Basis sog. \emph{Berechnungsmodelle} beschrieben \cite{ells:1997,ls:1998}. 
So werden Signalverarbeitungsalgorithmen oftmals mit Hilfe von \emph{Datenflussmodellen} \cite{lp:1995} beschrieben, w�hrend Protokolle oftmals durch \emph{endliche Zustandsautomaten} \cite{harel:1987} modelliert werden.
Beschr�nkte Berechnungsmodelle bilden die Grundlage zur effizienten Analyse und Implementierung einer Anwendung. 
SystemC \cite{sysc1,glms:2002} bietet die M�glichkeit, beliebige Berechnungsmodelle zu beschreiben. 
Hierin liegt aber auch das Problem, dass das zugrundeliegende (beschr�nkte) Berechnungsmodell nicht mehr erkannt werden kann.
\par
Um dieses Problem zumindest teilweise l�sen zu k�nnen, wurde am Lehrstuhl f�r Hardware-Software-Co-Design eine Aktorbibliothek mit dem Namen \emph{SysteMoC} entwickelt. 
Diese erlaubt es, Informationen �ber das zugrundeliegende Berechnungsmodell zu erkennen und zu analysieren.
Im Folgenden wird zun�chst die Idee der SysteMoC-Bibliothek vorgestellt, bevor auf die Verwendung dieser Bibliothek eingegangen wird. 
Schlie�lich wird noch beschrieben, wie es mit der SysteMoC-Bibliothek m�glich ist, das Zeitverhalten zu modellieren.
Hierzu kommt das am Lehrstuhl entwickelte \emph{Virtual Processing Components} Framework, welches bereits in der vorangegangenen Studie verwendet wurde, zum Einsatz.

Eine Anwendung wird oftmals in verschiedene Module zergliedert, welche meist von den einzelnen Entwicklern noch gut zu �berblicken sind.
Die eigentliche Komplexit�t einer Anwendung ergibt sich aber durch das Kommunikationsverhalten dieser Module miteinander.
Um diesem Ph�nomen zu begegnen ist eine systematische Beschreibung des Kommunikationsverhalten der Module von essentieller Bedeutung.
Zu diesem Zwecke wird die Anwendung in einzelne sog. \emph{Aktoren} zerlegt, welchen eine Kommunikation untereinander nur �ber sog. \emph{Kan�le} mittels \emph{Nachrichten} erlaubt ist.
Die zuvor genannten Module entsprechen dabei den Aktoren, welche noch einmal in \emph{Aktivierungsregeln} und \emph{Funktionalit�t} unterteilt werden.
Die Aktivierungsregeln kodieren dabei das Kommunikationsverhalten des Aktors und sind in Form eines endlichen Zustandsautomaten dargestellt.
Die Zust�nde dieses Zustandsautomaten werden im folgenden als \emph{Aktivierungszust�nde} bezeichnet.
Die Funktionalit�t setzt sich aus einer Menge von \emph{Aktionen} zusammen, welche die
eigentliche Datenverarbeitung durchf�hren und Nachrichten f�r die Kommunikation der Aktoren untereinander generieren.
Die Ausf�hrung dieser Aktionen kann den internen Zustand eines Aktors, im Folgenden als \emph{Funktionalit�tszustand}
bezeichnet, ver�ndern.
Die Aktivierungszust�nde und die Funktionalit�tszust�nde sind dabei zwei disjunkte Zustandsmengen, welche zusammen
die m�glichen \emph{Aktorzust�nde} ergeben.
Durch die vorgegebenen Syntax zur Kodierung der Aktivierungsregeln ist es m�glich,
die Zustandsautomaten aus den SysteMoC Aktoren zu extrahiert.
Diese extrahierten Aktivierungsregeln erlauben die Zuordnung des Aktors zu einem bestimmten Berechnungsmodell.
Des weiteren sind die hier verwendeten Prinzipien eng mit der Modellierungssprache \emph{FunState} (Functions driven by State machines) verwandt \cite{stgzet:2001}.
In \cite{strehl:2000} wurde gezeigt, wie man symbolische Techniken \cite{kropf:1999} verwenden kann, um FunState-Beschreibungen zu verifizieren und zu synthetisieren.
Die Bestimmung eines g�ltigen statischen Ablaufplans mit beschr�nktem Speicher f�r FunState-Beschreibungen beruht hierbei auf sog.\ \emph{regul�ren Zustandsmaschinen} \cite{tts:2000}.
Aufgrund der �hn\-lich\-keiten in den Konzepten von SysteMoC und FunState k�nnen die f�r FunState entwickelten Methoden auf die SysteMoC �bertragen werden.

% LocalWords:  SysteMoC
