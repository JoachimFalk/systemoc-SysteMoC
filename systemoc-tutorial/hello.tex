%%%%%%%%%%%%%%%%%%%%%%%%%%%%%%%%%%%%%%%%%%%%%%%%%%%%%%%%%%%%%%%%%%%%%%%%%%%%%%
%%%%%%%%%%%%%%%%%%%%%%%%%%%%%%%%%%%%%%%%%%%%%%%%%%%%%%%%%%%%%%%%%%%%%%%%%%%%%%
\begin{frame}
\mode<presentation>{\frametitle{\insertsubsection\ -- Objectives}}
\begin{itemize}
\item You will see a state-of-the-art ``Hello World'' application ...
\item ...  modeled in SysteMoC.
\end{itemize}
\end{frame}




%%%%%%%%%%%%%%%%%%%%%%%%%%%%%%%%%%%%%%%%%%%%%%%%%%%%%%%%%%%%%%%%%%%%%%%%%%%%%%
%%%%%%%%%%%%%%%%%%%%%%%%%%%%%%%%%%%%%%%%%%%%%%%%%%%%%%%%%%%%%%%%%%%%%%%%%%%%%%
 % fragile is mandatory for verbatim environments (lstlisting)
 % HACK: use singleslide for correct page numbering
 %  (page-counter+=2 else wise)
 %  this issue seems to be relevant for *pdflatex* only
 % NOTE: the intended usage for *singleslide* is to tell beamer that we use
 %   no overlays (animation)
\begin{frame}[fragile=singleslide]
\mode<presentation>{\frametitle{\insertsubsection}}
\begin{lstlisting}
//file hello.cpp
#include <iostream>
#include <systemoc/smoc_moc.hpp>

class HelloActor: public smoc_actor {
public:
  // actor constructor
  HelloActor(sc_module_name name)
    : smoc_actor(name, start) {
    // FSM definition:
    //  transition from start to end calling action src
    start = CALL(HelloActor::src) >> end;
  }
private:
  smoc_firing_state start, end;   // FSM states

  void src() {   // action
    std::cout << "Actor " << this->name() << " says:\n"
              << "Hello World" << std::endl;
  }
};
\end{lstlisting}
\end{frame}

%%%%%%%%%%%%%%%%%%%%%%%%%%%%%%%%%%%%%%%%%%%%%%%%%%%%%%%%%%%%%%%%%%%%%%%%%%%%%%
%%%%%%%%%%%%%%%%%%%%%%%%%%%%%%%%%%%%%%%%%%%%%%%%%%%%%%%%%%%%%%%%%%%%%%%%%%%%%%
\begin{frame}[fragile=singleslide]
\mode<presentation>{\frametitle{\insertsubsection}}
\begin{lstlisting}
//file hello.cpp cont'd
class HelloNetworkGraph: public smoc_graph {
public:
  // network graph constructor
  HelloNetworkGraph(sc_module_name name)
    : smoc_graph(name),
      helloActor("HelloActor") // create actor HelloActor
  { }
private:
  // actors
  HelloActor     helloActor;
};

int sc_main (int argc, char **argv) {
  // create network graph
  HelloNetworkGraph top("top");
  smoc_scheduler_top sched(top);

  sc_start();  // start simulation (SystemC)
  return 0;
}
\end{lstlisting}
\end{frame}

%%%%%%%%%%%%%%%%%%%%%%%%%%%%%%%%%%%%%%%%%%%%%%%%%%%%%%%%%%%%%%%%%%%%%%%%%%%%%%
%%%%%%%%%%%%%%%%%%%%%%%%%%%%%%%%%%%%%%%%%%%%%%%%%%%%%%%%%%%%%%%%%%%%%%%%%%%%%%
\begin{frame}[fragile=singleslide]
\mode<presentation>{\frametitle{\insertsubsection}}
\begin{itemize}
\item compile the source code
\item run simulation (OSCI SystemC: execute binary for simulation )
\begin{lstlisting}
./hello
\end{lstlisting}
\item simulation output
\begin{lstlisting}
             SystemC 2.2.0 --- Dec 15 2008 11:10:07
        Copyright (c) 1996-2006 by all Contributors
                    ALL RIGHTS RESERVED
Actor top.HelloActor says:
Hello World
SystemC: simulation stopped by user.
\end{lstlisting}
\end{itemize}
\end{frame}





