%%%%%%%%%%%%%%%%%%%%%%%%%%%%%%%%%%%%%%%%%%%%%%%%%%%%%%%%%%%%%%%%%%%%%%%%%%%%%%
%%%%%%%%%%%%%%%%%%%%%%%%%%%%%%%%%%%%%%%%%%%%%%%%%%%%%%%%%%%%%%%%%%%%%%%%%%%%%%
\begin{frame}
\mode<presentation>{\frametitle{\insertsubsection\ -- Objectives}}
\begin{itemize}
\item You will learn to ...
\item ... write and use guards.
\item ... write and use actions.
\end{itemize}
\end{frame}




%%%%%%%%%%%%%%%%%%%%%%%%%%%%%%%%%%%%%%%%%%%%%%%%%%%%%%%%%%%%%%%%%%%%%%%%%%%%%%
%%%%%%%%%%%%%%%%%%%%%%%%%%%%%%%%%%%%%%%%%%%%%%%%%%%%%%%%%%%%%%%%%%%%%%%%%%%%%%
\begin{frame}[fragile=singleslide]
\mode<presentation>{\frametitle{\insertsubsection\ -- SourceActor}}
\begin{lstlisting}
static const std::string MESSAGE = "0123456789";

class Source: public smoc_actor {
public:
  smoc_port_out<char> out;
  Source(sc_module_name name) : smoc_actor(name, start),
    count(0), size(MESSAGE.size()), message(MESSAGE) {
    start = 
      GUARD(Source::hasToken)  >>
      out(1)                   >>
      CALL(Source::src)        >> start;
  }
private:
  smoc_firing_state start;

  unsigned int count, size;  // variables (functional state)
  const std::string message; //

  bool hasToken() const{ return count<size; } // guard
  void src() { out[0] = message[count++]; }   // action
};

\end{lstlisting}
\end{frame}






%%%%%%%%%%%%%%%%%%%%%%%%%%%%%%%%%%%%%%%%%%%%%%%%%%%%%%%%%%%%%%%%%%%%%%%%%%%%%%
%%%%%%%%%%%%%%%%%%%%%%%%%%%%%%%%%%%%%%%%%%%%%%%%%%%%%%%%%%%%%%%%%%%%%%%%%%%%%%
\begin{frame}[fragile=singleslide]
\mode<presentation>{\frametitle{\insertsubsection\ -- Guards}}
\index{guard}
\index{GUARD@\lstinline{GUARD}}
\begin{itemize}
\item a guard is a \lstinline!const! member function returning a boolean value
\begin{lstlisting}
  bool hasToken() const{
    return count<size;
  } // guard
\end{lstlisting}
\item guards enable/disable transitions (true/false)
\item guards must (can) not change variable values or token in channels
\item refer to guards via \lstinline!GUARD(..)! macro 
\begin{lstlisting}
    start = 
      GUARD(Source::hasToken)  >>
      out(1)                   >>
      CALL(Source::src)        >> start;
\end{lstlisting}
\item use guards for control flow (see below)
\end{itemize}
\end{frame}







%%%%%%%%%%%%%%%%%%%%%%%%%%%%%%%%%%%%%%%%%%%%%%%%%%%%%%%%%%%%%%%%%%%%%%%%%%%%%%
%%%%%%%%%%%%%%%%%%%%%%%%%%%%%%%%%%%%%%%%%%%%%%%%%%%%%%%%%%%%%%%%%%%%%%%%%%%%%%
\begin{frame}[fragile=singleslide]
\mode<presentation>{\frametitle{\insertsubsection\ -- Variables}}
\index{variables}
\begin{itemize}
\item variables ...
\item ... are \lstinline!private! class member of an actor
\item ... can be used to store data
\item ... represent a functional state of an actor (in contrast to FSM state)
\begin{lstlisting}
  unsigned int count, size;
  const std::string message;
\end{lstlisting}
\end{itemize}
\end{frame}






%%%%%%%%%%%%%%%%%%%%%%%%%%%%%%%%%%%%%%%%%%%%%%%%%%%%%%%%%%%%%%%%%%%%%%%%%%%%%%
%%%%%%%%%%%%%%%%%%%%%%%%%%%%%%%%%%%%%%%%%%%%%%%%%%%%%%%%%%%%%%%%%%%%%%%%%%%%%%
\begin{frame}[fragile=singleslide]
\mode<presentation>{\frametitle{\insertsubsection\ -- Actions}}
\index{actions}
\begin{itemize}
\item actions ...
\item ... are used to read/write data on input/output ports
\item ... modify variables
\begin{lstlisting}
  void src() {
    out[0] = message[count++];
  }
\end{lstlisting}
\item guards access variables read-only (mandatory \lstinline!const! modifier)
\item actions are allowed to modify variables
\end{itemize}
\end{frame}






%%%%%%%%%%%%%%%%%%%%%%%%%%%%%%%%%%%%%%%%%%%%%%%%%%%%%%%%%%%%%%%%%%%%%%%%%%%%%%
%%%%%%%%%%%%%%%%%%%%%%%%%%%%%%%%%%%%%%%%%%%%%%%%%%%%%%%%%%%%%%%%%%%%%%%%%%%%%%
\begin{frame}[fragile=singleslide]
\mode<presentation>{\frametitle{\insertsubsection\ -- Multiple Access}}
\index{port}
\begin{itemize}
\item you can write/read data more than once (overwrite/re-read)
\item e.g. read input twice
\begin{lstlisting}
  void sink() {
    char squareInput = in[0] * in[0];

    char x = in[0];
    char y = in[0]; // re-read
    assert(x == y);
  }
\end{lstlisting}
\item e.g. write a default value first
\begin{lstlisting}
  void src() {
    out[0] = 'X';    // default
    if(count<size){
      out[0] = message[count++]; //overwrite
    }
  }
\end{lstlisting}
\end{itemize}
\end{frame}










%%%%%%%%%%%%%%%%%%%%%%%%%%%%%%%%%%%%%%%%%%%%%%%%%%%%%%%%%%%%%%%%%%%%%%%%%%%%%%
%%%%%%%%%%%%%%%%%%%%%%%%%%%%%%%%%%%%%%%%%%%%%%%%%%%%%%%%%%%%%%%%%%%%%%%%%%%%%%
\begin{frame}[fragile=singleslide]
\mode<presentation>{\frametitle{\insertsubsection}}
\begin{itemize}
\item simulation output (using Sink actor from previous example)
\begin{lstlisting}
top.Sink recv: "0"
top.Sink recv: "1"
top.Sink recv: "2"
top.Sink recv: "3"
top.Sink recv: "4"
top.Sink recv: "5"
top.Sink recv: "6"
top.Sink recv: "7"
top.Sink recv: "8"
top.Sink recv: "9"
SystemC: simulation stopped by user.
\end{lstlisting}
\item Source actor sends a finite number of characters only
\item simulation terminates when no actor can be activated
\end{itemize}
\end{frame}





%%%%%%%%%%%%%%%%%%%%%%%%%%%%%%%%%%%%%%%%%%%%%%%%%%%%%%%%%%%%%%%%%%%%%%%%%%%%%%
%%%%%%%%%%%%%%%%%%%%%%%%%%%%%%%%%%%%%%%%%%%%%%%%%%%%%%%%%%%%%%%%%%%%%%%%%%%%%%
%\begin{frame}[fragile=singleslide=singleslide]
%\mode<presentation>{\frametitle{\insertsubsection}}
%\begin{itemize}
%\item ...
%\end{itemize}
%\begin{lstlisting}
%
%\end{lstlisting}
%\end{frame}





