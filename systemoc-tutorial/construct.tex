%%%%%%%%%%%%%%%%%%%%%%%%%%%%%%%%%%%%%%%%%%%%%%%%%%%%%%%%%%%%%%%%%%%%%%%%%%%%%%
%%%%%%%%%%%%%%%%%%%%%%%%%%%%%%%%%%%%%%%%%%%%%%%%%%%%%%%%%%%%%%%%%%%%%%%%%%%%%%
\begin{frame}
\mode<presentation>{\frametitle{\insertsubsection\ -- Objectives}}
\begin{itemize}
\item You will learn how to dynamically construct the FSM of an actor...
\item ... by dynamically adding transitions to states and
\item ... by passing and returning states to and from functions
\end{itemize}
\end{frame}


%%%%%%%%%%%%%%%%%%%%%%%%%%%%%%%%%%%%%%%%%%%%%%%%%%%%%%%%%%%%%%%%%%%%%%%%%%%%%%
%%%%%%%%%%%%%%%%%%%%%%%%%%%%%%%%%%%%%%%%%%%%%%%%%%%%%%%%%%%%%%%%%%%%%%%%%%%%%%
\begin{frame}[fragile=singleslide]
\mode<presentation>{\frametitle{\insertsubsection}}
\begin{itemize}
\item Transitions can be added to states after some transitions have already been added:
\begin{lstlisting}
  smoc_firing_state a, b, c, d;
  
  // overwrite any transitions already added to a
  a =
      GUARD(Actor::guardA) >> CALL(Actor::actionA) >> a
    | GUARD(Actor::guardB) >> CALL(Actor::actionB) >> b;
  
  // append some more transitions to a
  a |= 
      GUARD(Actor::guardC) >> CALL(Actor::actionC) >> c
    | GUARD(Actor::guardD) >> CALL(Actor::actionD) >> d;  
\end{lstlisting}
\item Note that whereas \texttt{a = <transition list>} first removes all transitions from a and then
      adds \texttt{<transition list>} to a, \texttt{a |= <transition list>} simply appends \texttt{<transition list>}
      to a
\end{itemize}
\end{frame}

%%%%%%%%%%%%%%%%%%%%%%%%%%%%%%%%%%%%%%%%%%%%%%%%%%%%%%%%%%%%%%%%%%%%%%%%%%%%%%
%%%%%%%%%%%%%%%%%%%%%%%%%%%%%%%%%%%%%%%%%%%%%%%%%%%%%%%%%%%%%%%%%%%%%%%%%%%%%%
\begin{frame}[fragile=singleslide]
\mode<presentation>{\frametitle{\insertsubsection}}
\begin{itemize}
\item States can be passed to functions via pointers or references:
\begin{lstlisting}
  void a(smoc_firing_state& s);
  void b(smoc_firing_state::Ref s); // equivalent to a
  
  void c(const smoc_firing_state& s);
  void d(smoc_firing_state::ConstRef s); // equivalent to c

  void e(smoc_firing_state* a);
  void f(smoc_firing_state::Ptr a); // equivalent to e

  void g(const smoc_firing_state* a);
  void h(smoc_firing_state::ConstPtr a); // equivalent to g
\end{lstlisting}
\item Due to states being reference counted, using \texttt{::Ref}, \texttt{::ConstRef}, \texttt{::Ptr} and \texttt{::ConstPtr} is the preferred way
      of passing around references or pointers to states
\end{itemize}
\end{frame}

%%%%%%%%%%%%%%%%%%%%%%%%%%%%%%%%%%%%%%%%%%%%%%%%%%%%%%%%%%%%%%%%%%%%%%%%%%%%%%
%%%%%%%%%%%%%%%%%%%%%%%%%%%%%%%%%%%%%%%%%%%%%%%%%%%%%%%%%%%%%%%%%%%%%%%%%%%%%%
\begin{frame}[fragile=singleslide]
\mode<presentation>{\frametitle{\insertsubsection}}
\begin{itemize}
\item States can be returned from functions via pointers or references:
\begin{lstlisting}
  smoc_firing_state::Ptr a() {
    smoc_firing_state s;
    ...
    return &s;
  }
  
  smoc_firing_state::Ref b() {
    smoc_firing_state s;
    ...
    return s;
  }
\end{lstlisting}
\item Note that when using \texttt{::Ref}, \texttt{::ConstRef}, \texttt{::Ptr} or \texttt{::ConstPtr} as the return type of a function,
      returning a pointer or reference to a state declared locally inside the function is perfectly legal
\end{itemize}
\end{frame}
