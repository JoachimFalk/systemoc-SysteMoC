%%%%%%%%%%%%%%%%%%%%%%%%%%%%%%%%%%%%%%%%%%%%%%%%%%%%%%%%%%%%%%%%%%%%%%%%%%%%%%
%%%%%%%%%%%%%%%%%%%%%%%%%%%%%%%%%%%%%%%%%%%%%%%%%%%%%%%%%%%%%%%%%%%%%%%%%%%%%%
\begin{frame}
\mode<presentation>{\frametitle{\insertsubsection\ -- Objectives}}
\begin{itemize}
\item You will learn to ...
\item ... write time-triggered actors
\item ... create a graph for time-triggered actors
\item ... use signals with multiple receivers (multicast)
\item ... create a typical automotive (AUTOSAR-like) network
\end{itemize}
\end{frame}




%%%%%%%%%%%%%%%%%%%%%%%%%%%%%%%%%%%%%%%%%%%%%%%%%%%%%%%%%%%%%%%%%%%%%%%%%%%%%%
%%%%%%%%%%%%%%%%%%%%%%%%%%%%%%%%%%%%%%%%%%%%%%%%%%%%%%%%%%%%%%%%%%%%%%%%%%%%%%
\begin{frame}[fragile=singleslide]
\mode<presentation>{\frametitle{\insertsubsection\ -- Motivation}}
\index{periodic actor}
\index{CALL@\lstinline{CALL}}
Not everything can be modelled within a synchronous-reactive or data-flow environment. Some systems need special
 features and activation patterns.
\begin{itemize}
\item Especially automotive-applications are often triggered by timed events
	\begin{itemize}
	\item control loops are designed to run periodically
	\item activation is done via timer or the operating system
	\item synchronisation with time-triggered busses (e.g. FlexRay)
	\end{itemize}
\item messages in (automotive-)networks are often send periodically
\item more and more time-triggered-techniques are used
\item some tasks run pseudo-periodical with moving periods or jitter
\end{itemize}
\vspace{5mm}
It's worth to support time-triggered-functionality in SysteMoC natively!
\index{periodic actor|)}
\end{frame}




%%%%%%%%%%%%%%%%%%%%%%%%%%%%%%%%%%%%%%%%%%%%%%%%%%%%%%%%%%%%%%%%%%%%%%%%%%%%%%
%%%%%%%%%%%%%%%%%%%%%%%%%%%%%%%%%%%%%%%%%%%%%%%%%%%%%%%%%%%%%%%%%%%%%%%%%%%%%%
\begin{frame}[fragile=singleslide]
\mode<presentation>{\frametitle{\insertsubsection\ -- Periodic Actor}}
\index{periodic actor}
\index{CALL@\lstinline{CALL}}
You can create an actor, that will be activated in a fixed period of time
\begin{itemize}
\item the structure of a periodic actor is mostly equivalent to an ``normal'' actor,
 but there are some differences
\item it is available within header \lstinline{smoc_tt.hpp}
\begin{lstlisting}
#include <systemoc/smoc_tt.hpp>
\end{lstlisting}

\item a periodic actor has to inherit from base-class \lstinline{smoc_periodic_actor}
\begin{lstlisting}
class PActor: public smoc_periodic_actor {
\end{lstlisting}
\item the constructor has to set the period and an offset for the activation
\begin{lstlisting}
PActor(sc_module_name name, sc_time period, sc_time offset):
 smoc_periodic_actor(name, state_a, period, offset) {
   ...
}
\end{lstlisting}
\end{itemize}
\index{periodic actor|)}
\end{frame}




%%%%%%%%%%%%%%%%%%%%%%%%%%%%%%%%%%%%%%%%%%%%%%%%%%%%%%%%%%%%%%%%%%%%%%%%%%%%%%
%%%%%%%%%%%%%%%%%%%%%%%%%%%%%%%%%%%%%%%%%%%%%%%%%%%%%%%%%%%%%%%%%%%%%%%%%%%%%%
\begin{frame}[fragile=singleslide]
\mode<presentation>{\frametitle{\insertsubsection\ -- Special Remarks}}
\index{periodic actor}
\index{CALL@\lstinline{CALL}}
\begin{itemize}
\item offset and period must be determined carefully, they affect the system behaviour
\item reactivation of a time-tiggered actor during its execution is not allowed; execution is delayed to the next time-triggered activation
\item it must be guaranteed that an actor can read/write enough tokens from input/output-ports (e.g. by using a \lstinline{smoc_register}-channel)
\item to model a dynamic activation-jitter of a Task, a \lstinline{jitter} can be added. This factor specifies the time an actor-activation could jitter around the normal value.
\begin{lstlisting}
PActor(sc_module_name name, sc_time period, sc_time offset, float jitter):
 smoc_periodic_actor(name, state, period, offset, jitter){
   ...
}
\end{lstlisting}
\end{itemize}
\index{periodic actor|)}
\end{frame}




%%%%%%%%%%%%%%%%%%%%%%%%%%%%%%%%%%%%%%%%%%%%%%%%%%%%%%%%%%%%%%%%%%%%%%%%%%%%%%
%%%%%%%%%%%%%%%%%%%%%%%%%%%%%%%%%%%%%%%%%%%%%%%%%%%%%%%%%%%%%%%%%%%%%%%%%%%%%%
\begin{frame}[fragile=singleslide]
\mode<presentation>{\frametitle{\insertsubsection\ -- Time-triggered Graph}}
\index{periodic actor}
\index{CALL@\lstinline{CALL}}
\begin{itemize}
\item The time-triggered actors need a special environment, the \lstinline{smoc_graph_tt}.
\item This time-triggered graph can handle the time-triggered activations of the actors
\item but it also supports actors with activation by dynamic dataflow
\begin{lstlisting}
class TTNetworkGraph : public smoc_graph_tt {
   PActor pactor;
// network graph constructor
TTNetworkGraph ( sc_module_name name)
: smoc_graph_tt(name),
  pactor("PA", sc_time(5, SC_MS), sc_time(1,SC_MS))
  { }
};
\end{lstlisting}
\end{itemize}
\index{periodic actor|)}
\end{frame}




%%%%%%%%%%%%%%%%%%%%%%%%%%%%%%%%%%%%%%%%%%%%%%%%%%%%%%%%%%%%%%%%%%%%%%%%%%%%%%
%%%%%%%%%%%%%%%%%%%%%%%%%%%%%%%%%%%%%%%%%%%%%%%%%%%%%%%%%%%%%%%%%%%%%%%%%%%%%%
\begin{frame}[fragile=singleslide]
\mode<presentation>{\frametitle{\insertsubsection\ -- Multicast}}
\index{CALL@\lstinline{CALL}}
\begin{itemize}
\item Especially in automotive-networks, the use of shared media communication is common
\item a single signal could be received by multiple actors
\item so called \textit{data distribution semantic}
\item not realizable with direct point-to-point channels (e.g. smoc\_fifo)
\item especially in time-triggered systems, it's common to have a kind of under- and oversampling
\begin{itemize}
\item it's possible to have an initial value
\item a signal can be read more then one time
\item a signal can be updated before is was read or processed
\end{itemize}
\end{itemize}
\end{frame}




%%%%%%%%%%%%%%%%%%%%%%%%%%%%%%%%%%%%%%%%%%%%%%%%%%%%%%%%%%%%%%%%%%%%%%%%%%%%%%
%%%%%%%%%%%%%%%%%%%%%%%%%%%%%%%%%%%%%%%%%%%%%%%%%%%%%%%%%%%%%%%%%%%%%%%%%%%%%%
\begin{frame}[fragile=singleslide]
\mode<presentation>{\frametitle{\insertsubsection\ -- Register Channel}}
\index{register channel}
\index{CALL@\lstinline{CALL}}
\begin{itemize}
\item SysteMoC offers a specialized channel, the \lstinline{smoc_register<>}
\begin{lstlisting}
smoc_register<int> channel;
\end{lstlisting}
\item this channel has a register-semantic and could be connected to one sender and multiple receivers
\begin{lstlisting}
channel.connect(app1.out)  // sender
       .connect(app2.in)   // receiver
       .connect(app3.in);  // receiver
\end{lstlisting}
\item a possible problem can be caused by time-triggered actors. They can be activated despite not enough data on the channel,
 so it is possible that ``undefined'' data will be read. This must be avoided e.g. by a initial value.
\begin{lstlisting}
channel << 4711;
\end{lstlisting}
\end{itemize}
\index{register channel|)}
\end{frame}




%%%%%%%%%%%%%%%%%%%%%%%%%%%%%%%%%%%%%%%%%%%%%%%%%%%%%%%%%%%%%%%%%%%%%%%%%%%%%%
%%%%%%%%%%%%%%%%%%%%%%%%%%%%%%%%%%%%%%%%%%%%%%%%%%%%%%%%%%%%%%%%%%%%%%%%%%%%%%
\begin{frame}[fragile=singleslide]
\mode<presentation>{\frametitle{\insertsubsection\ -- Automotive Networks}}
\index{register channel}
\index{CALL@\lstinline{CALL}}
\begin{itemize}
\item Especially in automotive-networks, different communication-semantics are used
\item therefore SysteMoC supports both common techniques used in the AUTOSAR-Standard
\item the techniques can be realized by different kinds of channels
\begin{itemize}
\item[queued] means the communication between ports is handled like a FIFO-Buffer, so you should use a \lstinline{smoc_fifo<>}
\item[last-is-best] means the communication between ports is handled as a Register-Buffer, so the use of \lstinline{smoc_register<>} is recommended
\begin{itemize}
\item this is the default semantic in time-triggered networks
\end{itemize}
\end{itemize}
\end{itemize}
\index{register channel|)}
\end{frame}


